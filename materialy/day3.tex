% !TEX root = materialy.tex
\section{Funktor}
\emph{Funktory} to przekształcenia między kategoriami. Ściślej, funktor (\emph{kowariantny}) $\morph T{\mathcal B}{\mathcal C}$ to para ,,funkcji'' (obie oznaczane $T$). Jedna przekształca obiekty z $\mathcal B$ na obiekty na $\mathcal C$, a druga morfizmy z $\mathcal B$ na morfizmy w $\mathcal C$, tak żeby
\begin{itemize}
\item $T(\id B) = \id {T(B)}$ dla wszystkich obiektów $B$ z $\mathcal B$,
\item $T(f\circ\ g) = T(f) \circ\ T(g)$ dla wszystkich morfizmów $f$, $g$ w $B$ (dla których złożenie $f\circ g$ ma sens).
\end{itemize}
Funktor \emph{kontrawariantny} jest zdefiniowany analogiczne, zmieniamy tylko drugie wymaganie na
\begin{itemize}
	\item $T(f\  \circ\ g) = T(g)\ \circ\ T(f)$ dla wszystkich morfizmów $f$, $g$ w $\mathcal B$ (dla których złożenie $f\circ g$ ma sens).
\end{itemize}

\begin{example}
	Funktor identycznościowy dla kategorii $\mathcal B$, to funktor $\morph T{\mathcal B}{\mathcal B}$, który dla każdego obiektu i morfizmu zwraca go spowrotem.
\end{example}

\begin{example}
	Funktor zapominalski dla kategorii $\Top$, to funktor $\morph T\Top\Set$ który dla każdej przestrzeni topologicznej $(X,\  \mathcal T)$ zwraca  $X$ i nie zmienia morfizmów. (Każda funkcja ciągła jest przecież funkcją).
\end{example}

\begin{example}
	Zdefiniujemy funktory $\morph{F, G}\Set\Set$ w następujący sposób: dla każdego zbioru $X$ obydwa zwracają zbiór potęgowy $P(X)$. $F$ zamienia morfizmy na branie obrazów, a $G$ na branie przeciwobrazów. To znaczy dla $\morph fXY$:
	\begin{itemize}
		\item $F(f):\ P(X)\ \ni\ U\ \to\ f(U)\ \in\ P(Y)$
		\item $G(f):\ P(Y)\ \ni\ U\ \to\ f^{-1}(U)\ \in\ P(X)$
	\end{itemize}
\end{example}

\begin{problem}
  Pokaż, że funktory zdefiniowane w przykładach spełniają aksjomaty. Rozpoznaj, który jest funktorem kowariantnym, a który kontrawariantny.
\end{problem}

\begin{problem}
	Czy potrafisz podać inne przykłady funktorów?
\end{problem}

\section{Grupa fundamentalna}
\emph{Grupę  fundamentalną} przestrzeni $(X, p)$ z $\Top_*$ oznaczamy $\pi(X, p)$. Jej elementami są klasy abstrakcji przekształceń:
\[
	\morph f{\mathbb S}X
\]
spełniające warunek $f(1, 0) = p$, które są \emph{homotopijne} -- $f \sim g$ jeśli istnieje funkcja ciągła
\[
	\morph{F}{I \times \mathbb{S}} X
\]
taka, że $\ F(0,\ \cdot)\ = f$ i $F(1,\ \cdot)\ = g$.
(Wyobrażamy sobie, że zmieniając pierwszy parametr przeciągamy jedną pętelkę na drugą -- spójrz na tablicę).

Określamy \emph{mnożenie pętelek} $f$ i $g$ jako
\[
	(f*g)(x) = \begin{cases}
								f(2x) & \text{dla } x\in [0, 1/2]\\
								g(2x-1) & \text{dla } x\in [1/2, 1]
					 	\end{cases}
\]
(Jedziemy po pętelce $f$, a potem po pętelce $g$. Musimy poruszać się dwa razy szybciej niż zwykle).

\begin{problem}
	Pokaż, że mnożenie klas abstrakcji jest dobrze określone: $[f * g] = [f' * g']$ jeśli $[f]=[f']$ oraz $[g]=[g']$.

	Następnie pokaż, że klasa abstrakcji funkcji stałej $c(x)=p$ jest identycznością. Pokazując łączność mnożenia oraz istnienie odwrotności zauważ, że otrzymaliśmy strukturę grupy.
\end{problem}

\begin{problem}
	Pokaż, że otrzymaliśmy funktor kowariantny $\morph{\pi}{\Top_*}\Grp$. (W jaki sposób działa na morfizmach?)
\end{problem}

\begin{problem}
	Pokaż, że jeżeli istnieje $\morph fIX$ takie, że  $f(0)= p$ oraz $f(1)\ = q$ to $\pi(X,\ p)$ jest izomorficzne z $\pi(X,\ q)$.
\end{problem}

\begin{remark}
	W szczególności dla przestrzeni łukowo spójnych nie ma znaczenia jaki punkt wybierzemy, zawsze dostajeniemy ,,tę samą'' grupę (z dokładnością do izomorfizmu).
	W takich przypadkach piszemy po prostu $\pi(X)$.
\end{remark}

\begin{example}
	Obliczenia dla $\pi(\mathbb{R})$ i $\pi(\mathbb{S})$ (tablica).
\end{example}

\begin{problem}
	Obliczyć $\pi(\mathbb R^n)$ dla wszystkich $n\ge 0$.
\end{problem}

\begin{problem}
	Obliczyć $\pi(\mathbb R^2\setminus \{0\})$.
\end{problem}

\begin{problem}
	Udowodnij, że $\mathbb{R}^2\setminus \{0\}$ nie jest homeomorficzne z $\mathbb{R}^2$.
\end{problem}

\begin{remark}
	Przy pomocy analogicznych funktorów $\pi$ da się pokazać, że sfery $\mathbb S^n$ i $\mathbb S^m$ nie są homeomorficzne dla $n\neq m$. (Analogicznie dla przestrzeni Euklidesowych $\mathbb R^n$).
\end{remark}

\section{Funktor spójności}

Przestrzeni topologiczna $X$ jest \emph{niespójna} jeżeli istnieje niepusty i właściwy ($U\neq X$ i $U\neq \varnothing$) podzbiór $U\subset X$ taki, że jest jednocześnie otwarty i domknięty. Przestrzeń topologiczna jest \emph{spójna} kiedy nie jest niespójna.
\emph{Spójną składową]} nazywamy maksymalny (w sensie zawierania), spójny i niepusty podzbiór $X$.

\begin{problem}
Pokaż, że jeżeli $A$ jest spójnym podzbiorem $X$, a $U,\ V$ są otwartymi rozłącznymi pozdbiorami $X$, to:
$$A \subseteq U\cup V \implies A \subseteq U \vee A \subseteq V$$
\end{problem}


\begin{problem}
	Pokaż, że spójne składowe tworzą partycję $X$.
\end{problem}

\begin{problem}
	Pokaż, że pod morfizmami obrazy zbiorów spójnych są spójne.
\end{problem}

\begin{problem}
	Skonstruuj funktor przypisujący przestrzeniem topologicznym zbiór ich spójnych składowych. W jaki sposób przekształca morfizmy?
\end{problem}

\begin{problem}
	Pokaż, że $\mathbb{R}$ nie jest homeomorficzne z $\mathbb{R}^2$.
\end{problem}
