\section{Funktor}


\emph{Funktory} to przekształcenia między kategoriami. Ściślej, funktor (\emph{kowariantny}) $T: C\to B$ to para "funkcji" (obie oznaczane $T$). Jedna przekształca obiekty z $B$ na obiekty na $C$, a druga morfizmy z $B$ na morfizmy w $C$, tak żeby
\begin{itemize}
\item $T(1_b) = 1_{T(b)}$ dla wszystkich obiektów $b$ z $B$
\item $T(f\  \circ\ g) = T(f)\ \circ\ T(g)$ dla wszystkich morfizmów $f$, $g$ w $B$
\end{itemize}
Funktor \emph{kontrawariantny} jest zdefiniowany analogiczne, zmieniamy tylko drugie wymaganie na
\begin{itemize}
	\item $T(f\  \circ\ g) = T(g)\ \circ\ T(f)$ dla wszystkich morfizmów $f$, $g$ w $B$
\end{itemize}

\begin{example}
	Funktor identycznościowy dla kategorii $B$, to funktor $T:B\to B$ który dla każdego obiektu i morfizmu zwraca go spowrotem.
\end{example}

\begin{example}
	Funktor zapominalski dla kategorii $\Top$, to funktor $T:\Top\to \Set$ który dla każdej przestrzeni topologicznej $(X,\  \mathcal T)$ zwraca  $(X)$ i nie zmienia morfizmów.
\end{example}

\begin{example}
	Niech $F, G: \Set \to \Set$, to para funktorów $T:\Top\to \Set$ które dla  każdego zbioru $X$ zwracają zbiór potęgowy $P(X)$. $F$ zamienia morfizmy  na branie obrazów, a $G$ na branie przeciwobrazów. To znaczy dla $f:X\to Y$:
	\begin{itemize}
		\item 	$F(f):\ P(X)\ \ni\ U\ \to\ f(U)\ \in\ P(Y)$
		\item 	$G(f):\ P(Y)\ \ni\ U\ \to\ f^{-1}(U)\ \in\ P(X)$
	\end{itemize}

\end{example}

\prob{
  Pokaż, że funktory zdefiniowane w przykładach spełniają akjomaty. Rozpoznaj które są funktorami kowariantnymi a który kontrawariantnymi. Czy potrafisz podać jakieś inne funktory?
}

\section{Grupa fundamentalna}
\emph{Grupę  fundamentalną} przestrzeni $(X,\ p)$ z $\Top_*$ oznaczamy $\pi(X,\ p)$. Jej elementami są klasy abstrakcji przekształceń:
$$f:\ \mathbb{S}\ \to\ X\ :\ f(0)\ =\ f(2\pi)\ =\ p$$
które są \emph{homotopijne}:
$$f \sim g \iff \exists\ F:\ I \times \mathbb{S}\ \to\ X\ :\ F(0,\ \cdot)\ = f\ \wedge\ F(1,\ \cdot)\ = g$$
Wyborażamy sobie, że zmieniając pierwszy parametr przeciągamy jedną pętelkę na drugą (patrz tablica).
Strukturę grupy dodajemy następująco:
\begin{itemize}
	\item $\pi(X,\ p)\ \ni\ e\ =\ [x\ \mapsto\ p]$
	\item $[f]\ +\ [g]\ =\ \bigg[x\ \mapsto\ f(2x)\text{ kiedy }x \in [0,\ ^1/_2]\text{ oraz }g(2x-1)\text{ kiedy }x \in [^1/_2,\ 1]\bigg]$
\end{itemize}