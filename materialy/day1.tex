\section{Grupy}

\begin{prob}
  Pomyśl o liczbach całkowitych i dodawaniu lub o permutacjach zbioru ${0,1,2}$ i składaniu funkcji oraz uzupełnij definicję grupy:

  \emph{Grupa} to para $(G, *)$, gdzie $G$ jest zbiorem, a funkcja $*: G\times G\to G$ (będziemy pisać $g*h$ lub $gh$ zamiast $*(g,h)$) ma następujące własności:
  \begin{enumerate}
    \item istnieje element $e\in G$ taki, że $\manydots$ Będziemy nazywać go \emph{elementem neutralnym}.
    \item dla każdego $g\in G$ istnieje element $g^{-1}\in G$ taki, że $\manydots$. Nazywamy go
      \emph{elementem odwrotnym do $g$}.
      \item dla dowolnych $f,\, g,\, h\in G$ zachodzi równość $\manydots$.
    \end{enumerate}
\end{prob}

\vspace{-1cm}

\prob{
  Ile istnieje grup czteroelementowych?\footnote{Z dokładnością do izomorfizmu. Heurystycznie chodzi o grupy, które mają taką samą strukturę, np. $\Z$ oraz $2\Z$ wyglądają dokładnie tak samo, poza tym, że mają inaczej nazwane elementy.}
}

\prob{
  Pomyśl o izometriach płaszyczny (przekształceniach zachowujących odległości, jak symetrie, przesunięcia czy obroty). Co jest złożeniem dwóch symetrii? Co jest złożeniem
  symetrii i przesunięcia? Czy do każdej takiej izometrii umiemy łatwo podać izometrię odwrotną?
}

\prob{
  Pokaż, że każda grupa izomorficzna jest do grupy permutacji jakiegoś zbioru.
}

\section{Topologia}
\subsection{Przestrzenie metryczne}
Czasami mamy na zbiorze zadaną \emph{metrykę} $d:X\times X\to \R$ o własnościach:

\begin{enumerate}
  \item $d(x,y)=0\Leftrightarrow x=y$
  \item $d(x,y)=d(y,x)$ dla wszystkich $x,\, y\in X$
  \item $d(x,y) \le d(x,k) + d(k, x)$ dla wszystkich $x,\, y,\, k\in X$
\end{enumerate}

W takim wypadku definiujemy \emph{kulę} o środku w $x$ i promieniu $r>0$:
$$B(x,\, r)=\{y\in X : d(x,y) < r\}$$

\prob{
  Narysuj kule w przestrzeniach:

  \begin{enumerate}
    \item $X,~d(x,y)=1$ dla $x\neq y$
    \item $\R, ~d(x_1,x_2)=|x_1-x_2|$
    \item $\R^2,~d(x_1, y_1, x_2, y_2)=\sqrt{(x_1-x_2)^2+(y_1-y_2)^2}$
    \item $\R^2,~d(x_1, y_1, x_2, y_2) = |x_1-x_2| + |y_1-y_2|$
    \item $\R^2,~d(x_1, y_1, x_2, y_2) = \text{max}(|x_1-x_2|, |y_1-y_2|)$
  \end{enumerate}
}

\subsection{Przestrzenie topologiczne}
\emph{Przestrzenią topologiczną} nazywamy parę $(X, \mathcal T)$, gdzie $X$ jest zbiorem, a $\mathcal T$ jest rodziną podzbiorów $X$ o następujących własnościach:
\begin{enumerate}
  \item $\varnothing,\, X\in \mathcal T$
  \item jeśli $A_i\in \mathcal T$ dla $i\in I$, to ich suma też należy do $\mathcal T$: $\bigcup_{i\in I}A_i\in \mathcal T$
  \item jeśli $A,\, B\in \mathcal T$, to $A\cap B\in\mathcal T$
\end{enumerate}
Zbiór $\mathcal T$ nazywamy \emph{topologią}, a jego elementy - \emph{zbiorami otwartymi}.


\begin{prob}
  Rozważmy przestrzeń metryczną $(X,\, d)$. Mówimy, że zbiór $A\subseteq X$ jest otwarty jeśli dla każdego $a\in A$ istnieje $r_a>0$ takie, że
  $$B(a,\, r_a)\subseteq A.$$
  Pokaż, że tak zdefiniowane zbiory otwarte rzeczywiście zadają topologię.
\end{prob}

\begin{prob}
  Jak wyglądają zbiory otwarte topologii wyznaczonych przez odległość bezwzględną na zbiorach $\R,\, \Z,\, \Q$?
\end{prob}

\begin{example}
  Topologia "rzeka", prosta z dwoma początkami.
\end{example}

\prob{
  Ile jest topologii na zbiorze $\{0,1,2\}$? (Z dokładnością do homeomorfizmu.)
}

\prob{
  Niech $X$ będzie dowolnym zbiorem, a $\mathcal T$ będzie rodziną zawierającą zbiór pusty oraz dopełnienia wszystkich skończonych podzbiorów $X$. Pokaż, że $\mathcal T$ jest topologią.
}

\prob{
  Rozważmy rodzinę zespolonych wielomianów
  $$S=\{w_i : i\in I\},~w_i: \C \ni z \mapsto a_nz^n+\dots+a_1z+a_0\in \C.$$
  Afiniczną rozmaitością algebraiczną $V(S)$ będziemy oznaczać zbiór liczb zespolonych, na których zerują się wszystkie wielomiany rodziny $S$.
  Pokaż, że dopełnienia wszystkich afinicznych rozmaitości algebraicznych tworzą topologię na $\C$ (nazywamy ją topologią Zariskiego).
}

\section{Kategorie}
\begin{prob}
  Pomyśl o zbiorach i funkcjach lub grupach i homomorfizmach oraz uzupełnij definicję kategorii:

  \emph{Kategorią} nazywamy kolekcję obiektów $A, B, C\dots$, taką, że dla każdych dwóch obiektów $A, B$ istnieje zbiór $\Hom(A, B)$ \textit{morfizmów} z $A$ do $B$, które można składać symbolem $\circ$. Mają one następujące własności:
  \begin{enumerate}
    \item Jeśli $f:A\to B$ oraz $g: B\to C$, to $\manydots$.
    \item Dla dowolnych trzech morfizmów $f:A\to B, g:B\to C, h:C\to D$ mamy równość: $\manydots$
    \item Dla każdego obiektu $X$ istnieje morfizm $\Id_X: X\to X$, taki że $\manydots$.
\end{enumerate}
\end{prob}

\vspace{-1cm}

\prob{
  Pokaż, że $\Set,~\Top,~\Grp,~\Set_*,~\Top_*$ są kategoriami.
}

\example{
  Kategoria z trzema morfizmami.
}
