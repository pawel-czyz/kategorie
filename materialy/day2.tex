% !TEX root = materialy.tex
\subsection{Epi/mono/izo-morfizmy}
Wprowadzimy pewne określenia morfizmów. Otóż niech $f:A\to B$ będzie morfizmem.

\begin{enumerate}
  \item $f$ jest \emph{epimorfizmem} jeśli dla dowolnego obiektu $C$ oraz morfizmów $g, g': B\to C$ z równości $g\circ f= g'\circ f$ wynika, że $g=g'$.
  \item $f$ is \emph{monomorfizmem} jeśli dla dowolnego obiektu $C$ oraz morfizmów $g, g': X\to A$ z równości $f\circ g= f\circ g'$ wynika, że
  $g=g'$.
  \item $f$ jest \emph{izomorfizmem} jeśli istnieje $f':B\to A$ takie, że $f\circ f'=\Id_B$ oraz $f'\circ f=\Id_A$
\end{enumerate}

\prob{
  Czym są epi/mono/izo-morfizmy w $\Set$?
}

\prob{
  Scharakteryzuj (nie musi być formalnie) powyższe morfizmy w kategoriach $\Top$ i $\Grp$.
}

\prob{
  Pokaż, że złożenie epi/mono/izo-morfizmów jest epi/mono/izo-morfizmem.
}

\subsection{Podobiekt}
\textit{Podobiektem} obiektu $A$ będziemy nazywać monomorfizm $g: B\to A$. Rozważając jednak kategorię, w której obiekty są pewnymi zbiorami, a morfizmy funkcjami, możemy
myśleć o obrazie $g$ (czyli zbiorze z dodatkową strukturą) jako o podobiekcie (np. podobiektem zbioru w kategorii $\Set$ jest podzbiór).

\prob{
  Pokaż, że podobiekt podobiektu jest podobiektem.
}

\prob{
  Scharakteryzuj podobiekty w $\Set,~\Top,~\Grp$. Dlaczego w $\Top$ istnieje kilka topologii jakie można nadać podzbiorowi aby był nadal podobiektem?
}

\subsection{Produkt}
Rozważmy dwa obiekty $A, B$ jakiejś kategorii. Ich \emph{produktem} będziemy nazywać
trójkę $(A\times B, \pi_A, \pi_B)$ taką, że:
\begin{itemize}
  \item $A\times B$ jest obiektem
  \item $\pi_A: A\times B\to A, \pi_B:A\times B\to B$ to morfizmy
  \item jeśli weźmiemy dowolny obiekt $X$ oraz morfizmy $f_A: X\to A, f_B: X\to B$ to
    istnieje \textit{dokładnie jeden} morfizm $f:X\to A\times B$ taki, że diagram \ref{fig:produkt} komutuje (czyli $f_A=\pi_A\circ f$, $f_B=\pi_B\circ f$).
\end{itemize}

\begin{figure}
  \centering
  \begin{tikzcd}
    & X \arrow[rd, "f_B"] \arrow[ld, "f_A"] \arrow[d, "f"] &\\
    A & \arrow[l, "\pi_A"] A\times B \arrow[r, "\pi_B"] & B
  \end{tikzcd}

  \caption{Produkt - definicja}
  \label{fig:produkt}
\end{figure}

\begin{prob}
  Pokaż, że produkt, jeśli istnieje, jest unikatowy z dokładnością do izomorfizmu.

  Podpowiedź: niech $(\tilde{A\times B}, \tilde \pi_A, \tilde \pi_B)$ będzie drugim produktem. Wtedy diagram 2 komutuje.
\end{prob}

\begin{figure}
  \centering
  \begin{tikzcd}
    & A\times B \arrow[rd, "\pi_B"] \arrow[ld, "\pi_A"] \arrow[d, "f"] &\\
    A & \arrow[l, "\tilde \pi_A"] A\tilde \times B \arrow[r, "\tilde \pi_B"] \arrow[d, "g"] & B\\
    & A\times B \arrow[ru, "\pi_B"] \arrow[lu, "\pi_A"] &
  \end{tikzcd}

  \caption{Produkt - unikatowość}
  \label{fig:produkt}
\end{figure}

\prob{
  Czym jest produkt w $\Set$ i $\Grp$? W jaki sposób skonstruować go w $\Top$?
}

\prob{
  Czy produkt zawsze musi istnieć?
}
