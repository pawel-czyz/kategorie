% !TEX root = materialy.tex
\subsection{Epi-, mono- oraz izomorfizmy}
Wprowadzimy pewne określenia morfizmów. Otóż niech $\morph fAB$ będzie morfizmem.

\begin{enumerate}
  \item $f$ jest \emph{epimorfizmem} jeśli dla dowolnego obiektu $C$ oraz morfizmów $\morph{g, g'}BC$ z równości $g\circ f= g'\circ f$ wynika $g=g'$.
  \item $f$ is \emph{monomorfizmem} jeśli dla dowolnego obiektu $C$ oraz morfizmów $\morph{g, g'}XA$ z równości $f\circ g= f\circ g'$ wynika $g=g'$.
  \item $f$ jest \emph{izomorfizmem} jeśli istnieje $\morph{f'}BA$ takie, że $f\circ f'=\id B$ oraz $f'\circ f=\id A$
\end{enumerate}

\begin{problem}
  Czym są epi-, mono- oraz izomorfizmy w $\Set$?
\end{problem}

\begin{problem}
  Scharakteryzuj (nie musi być formalnie) powyższe morfizmy w kategoriach $\Top$ i $\Grp$.
\end{problem}

\begin{problem}
  Pokaż, że złożenie epimorfizmów jest epimorfizmem. (Analogicznie dla monomorfizmów i izomorfizmów).
\end{problem}

\subsection{Podobiekt}
\textit{Podobiektem} obiektu $A$ będziemy nazywać monomorfizm $\morph gBA$. Rozważając jednak kategorię, w której obiekty są pewnymi zbiorami, a morfizmy funkcjami, możemy
myśleć o obrazie $g$ (czyli zbiorze z dodatkową strukturą) jako o podobiekcie (np. podobiektem zbioru w kategorii $\Set$ jest podzbiór).

\begin{problem}
  Pokaż, że podobiekt podobiektu jest podobiektem.
\end{problem}

\begin{problem}
  Scharakteryzuj podobiekty w $\Set,~\Top,~\Grp$. Dlaczego w $\Top$ istnieje kilka topologii jakie można nadać podzbiorowi aby był nadal podobiektem?
\end{problem}

\subsection{Produkt}
Rozważmy dwa obiekty $A, B$ jakiejś kategorii. Ich \emph{produktem} będziemy nazywać trójkę $(A\times B, \pi_A, \pi_B)$ taką, że:
\begin{itemize}
  \item $A\times B$ jest obiektem,
  \item $\morph{\pi_A}{A\times B}A,~\morph{\pi_B}{A\times B}B$ to morfizmy,
  \item jeśli weźmiemy dowolny obiekt $X$ oraz morfizmy $\morph{f_A}XA$ i $\morph {f_B}XB$ to istnieje \textit{dokładnie jeden} morfizm $\morph fX{A\times B}$ taki, że diagram

    \begin{diagram}
      & X \arrow[rd, "f_B"] \arrow[ld, "f_A", swap] \arrow[d, "f"] &\\
      A & \arrow[l, "\pi_A", swap] A\times B \arrow[r, "\pi_B"] & B
    \end{diagram}

  jest przemienny (czyli $f_A=\pi_A\circ f$, $f_B=\pi_B\circ f$).
\end{itemize}

\begin{problem}
  Pokaż, że produkt, jeśli istnieje, jest unikatowy z dokładnością do izomorfizmu.

  Podpowiedź: niech $(P, p_A, p_B)$ będzie drugim produktem. Przeanalizuj diagram
  \begin{diagram}
  & A\times B \arrow[rd, "\pi_B"] \arrow[ld, "\pi_A", swap] \arrow[d, "f"] &\\
  A & \arrow[l, "p_A", swap] P \arrow[r, "p_B"] \arrow[d, "g"] & B\\
  & A\times B \arrow[ru, "\pi_B", swap] \arrow[lu, "\pi_A"] &
  \end{diagram}
\end{problem}

\begin{problem}
  Czym jest produkt w $\Set$ i $\Grp$? W jaki sposób skonstruować go w $\Top$?
\end{problem}

\begin{problem}
  Czy produkt zawsze musi istnieć?
\end{problem}
