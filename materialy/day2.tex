% !TEX root = materialy.tex
Wprowadzimy pewne określenia morfizmów. Otóż niech $f:A\to B$ będzie morfizmem.

\begin{enumerate}
  \item $f$ jest \emph{epimorfizmem} jeśli dla dowolnego obiektu $C$ oraz morfizmów $g, g': B\to C$ z równości $g\circ f= g'\circ f$ wynika, że $g=g'$.
  \item $f$ is \emph{monomorfizmem} jeśli dla dowolnego obiektu $C$ oraz morfizmów $g, g': X\to A$ z równości $f\circ g= f\circ g'$ wynika, że
  $g=g'$.
  \item $f$ jest \emph{izomorfizmem} jeśli istnieje $f':B\to A$ takie, że $f\circ f'=\Id_B$ oraz $f'\circ f=\Id_A$
\end{enumerate}

\prob{
  Czym są powyższe mono/epi/izo-morfizmy w $\Set$?
}

\prob{
  Scharakteryzuj (nie musi być formalnie) powyższe morfizmy w kategoriach $\Top$ i $\Grp$.
}

\prob{
  Pokaż, że złożenie mono/epi/izo-morfizmów jest mono/epi/izo-morfizmem.
}

\textit{Podobiektem} obiektu $A$ będziemy nazywać monomorfizm $g: B\to A$. Rozważając jednak kategorię, w której obiekty są pewnymi zbiorami, a morfizmy funkcjami, możemy
myśleć o obrazie $g$ jako o podobiekcie (np. podzbiór można utożsamiać z tym podzbiorem).

\prob{
  Pokaż, że podobiekt podobiektu jest podobiektem.
}

\prob{
  Scharakteryzuj podobiekty w $\Set, \Top, \Grp$. Dlaczego w $\Top$ istnieje kilka
  topologii jakie można nadać podzbiorowi aby był nadal podobiektem?
}
