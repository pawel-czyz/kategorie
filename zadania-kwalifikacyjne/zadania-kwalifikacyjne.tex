\documentclass{article}

\usepackage{graphicx}
\usepackage[bottom]{footmisc}
\usepackage{listings}
\usepackage{amsmath}
\usepackage{amssymb}
\usepackage{polski}
\usepackage[T1]{fontenc}
\usepackage[utf8]{inputenc}
\usepackage{mathtools}
\usepackage{tikz-cd}

\newcounter{itemnum}
\newenvironment{prob}
{\stepcounter{itemnum}
\paragraph*{\arabic{itemnum}.}}
{}

\newcommand{\Hom}{\text{Hom}}
\newcommand{\Id}{\text{Id}}
\newcommand{\Set}{\textbf{Set}}

\begin{document}
\section{Wstęp}
  Celem tych zadań jest oswojenie Cię z podstawowymi pojęciami teorii kategorii. Nie musisz zrobić wszystkich, jednak szczególnie do tego zachęcamy - warsztaty będą polegać głównie na rozwiązywaniu zadań przy tablicy.

  Żadna dodatkowa literatura nie powinna być potrzebna do rozwiązania tych zadań, jednak w razie problemów z zadaniami, sugestii zmian lub chęci zobaczenia określonych kategorii na warsztatach, zachęcamy do kontaktu.

\section{Kategorie}
\emph{Kategorią} nazywamy kolekcję\footnote{Można myśleć o tym jak o zbiorze, do którego można włożyć dowolnie dużo elementów. Zainteresowanych jak uniknąć paradoksu zbioru wszystkich zbiorów odsyłamy do teorii klas Morse'a-Kelleya.} obiektów $A, B, C\dots$, taką, że dla każdych dwóch obiektów $A, B$ istnieje zbiór $\Hom(A, B)$ \textit{morfizmów} z $A$ do $B$. Na ogół piszemy $f:A\to B$ zamiast $f\in \Hom(A,B)$. Zakładamy też, że istnieje operacja składania morfizmów $\circ$ posiadająca następujące własności:
\begin{enumerate}
  \item Jeśli $f:A\to B$ oraz $g: B\to C$, to istnieje morfizm $g\circ f: A\to C$.
  \item Dla dowolnych trzech morfizmów $f:A\to B, g:B\to C, h:C\to D$ mamy równość: $(h\circ g)\circ f = h\circ(g\circ f).$
  \item Dla każdego obiektu $X$ istnieje morfizm $\Id_X: X\to X$, taki że dla każdego morfizmu $f:A\to B$ mamy $\Id_B\circ f=f=f\circ\Id_A$.
\end{enumerate}

\begin{prob}
  Niech obiektami będą zbiory, morfizmami funkcje, a operacją $\circ$ złożenie funkcji\footnote{Cóż za niezwykły zbieg okoliczności - to ten sam symbol!}. Pokaż, że
  jest to kategoria dowodząc, że:
  \begin{enumerate}
    \item dla dowolnych zbiorów $A, B$, wszystkie funkcje z $A$ do $B$ tworzą zbiór (nazywany $\Hom(A,B)$).
    \item dla dowolnych trzech funkcji $f:A\to B, g:B\to C, h:C\to D$ mamy równość: $(h\circ g)\circ f = h\circ(g\circ f).$
    \item dla każdego zbioru $X$ znajdź funkcję\footnote{Ciekawe jak się może nazywać...} $\Id_X$ taką, że dla dowolnej $f:A\to B$ mamy $\Id_B\circ f=f=f\circ\Id_A$.
  \end{enumerate}
  Kategorię tę nazywamy $\Set$.
\end{prob}

\section{Morfizmy}
\subsection{Epimorfizmy}
Niech $f:A\to B$ będzie morfizmem. Mówimy, że $f$ is \emph{epimorfizmem} jeśli dla dowolnego obiektu $C$ oraz morfizmów $g, g': B\to C$ z równości $g\circ f= g'\circ f$ wynika, że
$g=g'$.

\begin{prob}
  Pokaż, że jeśli $f:A\to B$ oraz $g: B\to C$ są epimorfizmami, to $g\circ f$ też jest epimorfizmem.
\end{prob}

\begin{prob}
  Pokaż, że w kategorii $\Set$ epimorfizmy to dokładnie surjekcje (każdy epimorfizm jest surjekcją, a każda surjekcja epimorfizmem).
\end{prob}


Jako, że obiektami kategorii nie muszą być zbiory, pojęcie "funkcji na" w ogólności może nie mieć sensu, dlatego zdanie "każdy epimorfizm to surjekcja" nie musi mieć sensu. Okazuje się, że

\begin{prob}
  Skonstruuj kategorię, której obiektami są pewne zbiory (tak więc pojęcie surjekcji ma sens), natomiast istnieje epimorfizm niebędący surjekcją.
  % Podpowiedź: wystarczą dwa obiekty.
\end{prob}

\subsection{Monomorfizmy}
Niech $f:A\to B$ będzie morfizmem. Mówimy, że $f$ is \emph{monomorfizmem} jeśli dla dowolnego obiektu $C$ oraz morfizmów $g, g': X\to A$ z równości $f\circ g= f\circ g'$ wynika, że
$g=g'$.

\begin{prob}
  Pokaż, że jeśli $f:A\to B$ oraz $g: B\to C$ są monomorfizmami, to $g\circ f$ też jest monomorfizmem.
\end{prob}

\begin{prob}
  Pokaż, że w kategorii $\Set$ monomorfizmy to dokładnie iniekcje (każdy monomorfizm jest iniekcją, a każda iniekcja monomorfizmem).
\end{prob}

\begin{prob}
  Skonstruuj kategorię, której obiektami są pewne zbiory (tak więc pojęcie surjekcji ma sens), natomiast istnieje monomorfizm niebędący iniekcją.
\end{prob}

\subsection{Izomorfizmy}
Niech $f:A\to B$ będzie morfizmem. Jeśli istnieje $f':B\to A$ takie, że $f\circ f'=\Id_B$ oraz $f'\circ f=\Id_A$, to $f$ nazywamy \emph{izomorfizmem}.


\begin{prob}
  \emph{Będziemy wykorzystywać izomorfizmy do utożsamiania pewnych przestrzeni. W tym celu przyda nam się własności podobne do występujących w definicji relacji równoważności \footnote{Nazwanie "dokładności do izomorfizmu" relacją równoważności jest kuszące, natomiast relacje równoważności określone są na \textit{zbiorach}. Kolekcja obiektów
  kategorii na ogół jest za duża do bycia zbiorem.}}.
  Pokaż, że:
  \begin{enumerate}
    \item dla każdego obiektu $A$ istnieje izomorfizm z $A$ do $A$.
    \item jeśli istnieje izomorfizm z $A$ do $B$, to istnieje też izomorfizm z $B$ do $A$.
    \item jeśli istnieje izomorfizm z $A$ do $B$ oraz z $B$ do $C$, to istnieje też izomorfizm z $A$ do $C$.
  \end{enumerate}
\end{prob}

\begin{prob}
  \emph{Izomorfizmy dzielą własności, które już wykazaliśmy.}
  \begin{enumerate}
  \item Pokaż, że każdy izomorfizm jest monomorfizmem oraz epimorfizmem.
  \item Skonstruuj kategorię, dowodzącą, że twierdzenie odwrotne nie jest prawdziwe (czyli, że istnieje morfizm będący epimorfizmem i monomorfizmem, natomiast nie izomorfizmem).
  \end{enumerate}
\end{prob}

\begin{prob}
  Czym są izomorfizmy w kategorii $\Set$?
\end{prob}

Powodzenia!
\end{document}
