\documentclass{article}

\usepackage{graphicx}
\usepackage[bottom]{footmisc}
\usepackage{listings}
\usepackage{amsmath}
\usepackage{amssymb}
\usepackage{polski}
\usepackage[T1]{fontenc}
\usepackage[utf8]{inputenc}
\usepackage{mathtools}
\usepackage{tikz-cd}

\newcounter{itemnum}
\newenvironment{prob}
{\stepcounter{itemnum}
\paragraph*{\arabic{itemnum}.}}
{~\newline}

\newcommand{\Hom}{\text{Hom}}
\newcommand{\Id}{\text{Id}}
\newcommand{\Set}{\textbf{Set}}
\newcommand{\Setstar}{\textbf{Set}_*}
\newcommand{\Top}{\textbf{Top}}
\newcommand{\Grp}{\textbf{Grp}}

\begin{document}
\section{Wstęp}
  Celem tych zadań jest oswojenie Cię z pojęciami, z których będziemy korzystać na warsztatach - zatem spróbuj rozwiązać wszystkie zadania, nabyta wiedza okaże się pomocna!

  Żadna dodatkowa literatura nie powinna być potrzebna do rozwiązania tych zadań, jednak w razie problemów z zadaniami, sugestii zmian lub chęci zobaczenia określonych kategorii na warsztatach, zachęcamy do kontaktu.

  Obok każdego zadania znajduje się liczba punktów, jakie planujemy za nie przyznać. Nie wykluczamy jednak zmian w punktacji. Za szczególnie eleganckie rozwiązanie, którego nie przewidzieliśmy, będziemy przyznawać dodatkowe punkty, wedle własnego poczucia estetyki.

\section{Kategorie}
\emph{Kategorią} nazywamy kolekcję
\footnote{Można myśleć o tym jak o zbiorze, do którego można włożyć \emph{dowolnie} dużo elementów. Zainteresowanych jak uniknąć paradoksu zbioru wszystkich zbiorów odsyłamy do teorii klas Morse'a-Kelleya.}
obiektów $A, B, C\dots$, taką, że dla każdych dwóch obiektów $A, B$ istnieje zbiór $\Hom(A, B)$ \textit{morfizmów} z $A$ do $B$. Na ogół piszemy $f:A\to B$ zamiast $f\in \Hom(A,B)$. Zakładamy też, że istnieje operacja składania morfizmów $\circ$ posiadająca następujące własności:
\begin{enumerate}
  \item Jeśli $f:A\to B$ oraz $g: B\to C$, to istnieje morfizm $g\circ f: A\to C$.
  \item Dla dowolnych trzech morfizmów $f:A\to B, g:B\to C, h:C\to D$ mamy równość: $(h\circ g)\circ f = h\circ(g\circ f).$
  \item Dla każdego obiektu $X$ istnieje morfizm $\Id_X: X\to X$, taki że dla każdego morfizmu $f:A\to B$ mamy $\Id_B\circ f=f=f\circ\Id_A$.
\end{enumerate}

\begin{prob}
  [8] Niech obiektami będą zbiory, morfizmami funkcje, a operacją $\circ$ złożenie funkcji\footnote{Cóż za niezwykły zbieg okoliczności - to ten sam symbol!}. Pokaż, że
  jest to kategoria dowodząc, że:
  \begin{enumerate}
    \item dla dowolnych zbiorów $A, B$, wszystkie funkcje z $A$ do $B$ tworzą zbiór (nazywany $\Hom(A,B)$).
    \item dla dowolnych trzech funkcji $f:A\to B, g:B\to C, h:C\to D$ mamy równość: $(h\circ g)\circ f = h\circ(g\circ f).$
    \item dla każdego zbioru $X$ znajdź funkcję\footnote{Ciekawe jak się może nazywać...} $\Id_X$ taką, że dla dowolnej $f:A\to B$ mamy $\Id_B\circ f=f=f\circ\Id_A$.
  \end{enumerate}
  Kategorię tę nazywamy $\Set$.
\end{prob}

\section{Morfizmy}
\subsection{Epimorfizmy}
Niech $f:A\to B$ będzie morfizmem. Mówimy, że $f$ is \emph{epimorfizmem} jeśli dla dowolnego obiektu $C$ oraz morfizmów $g, g': B\to C$ z równości $g\circ f= g'\circ f$ wynika, że
$g=g'$.

\begin{prob}
  [4] Pokaż, że jeśli $f:A\to B$ oraz $g: B\to C$ są epimorfizmami, to $g\circ f$ też jest epimorfizmem.
\end{prob}

\begin{prob}
  [4] Pokaż, że w kategorii $\Set$ epimorfizmy to dokładnie surjekcje (każdy epimorfizm jest surjekcją, a każda surjekcja epimorfizmem).
\end{prob}

Jako, że obiektami kategorii nie muszą być zbiory, pojęcie "funkcji na" w ogólności może nie mieć sensu, dlatego zdanie "każdy epimorfizm to surjekcja" nie musi mieć sensu. Okazuje się, że

\begin{prob}
  [5] Skonstruuj kategorię, której obiektami są pewne zbiory (tak więc pojęcie surjekcji ma sens), natomiast istnieje epimorfizm niebędący surjekcją.
  % Podpowiedź: wystarczą dwa obiekty.
\end{prob}

\subsection{Monomorfizmy}
Niech $f:A\to B$ będzie morfizmem. Mówimy, że $f$ is \emph{monomorfizmem} jeśli dla dowolnego obiektu $C$ oraz morfizmów $g, g': X\to A$ z równości $f\circ g= f\circ g'$ wynika, że
$g=g'$.

\begin{prob}
  [1] Pokaż, że jeśli $f:A\to B$ oraz $g: B\to C$ są monomorfizmami, to $g\circ f$ też jest monomorfizmem.
\end{prob}

\begin{prob}
  [1] Pokaż, że w kategorii $\Set$ monomorfizmy to dokładnie iniekcje (każdy monomorfizm jest iniekcją, a każda iniekcja monomorfizmem).
\end{prob}

\begin{prob}
  [1] Skonstruuj kategorię, której obiektami są pewne zbiory (tak więc pojęcie surjekcji ma sens), natomiast istnieje monomorfizm niebędący iniekcją.
\end{prob}

\subsection{Izomorfizmy}
Niech $f:A\to B$ będzie morfizmem. Jeśli istnieje $f':B\to A$ takie, że $f\circ f'=\Id_B$ oraz $f'\circ f=\Id_A$, to $f$ nazywamy \emph{izomorfizmem}.

\begin{prob}
  \emph{Będziemy wykorzystywać izomorfizmy do utożsamiania pewnych przestrzeni. W tym celu przyda nam się własności podobne do występujących w definicji relacji równoważności \footnote{Nazwanie "dokładności do izomorfizmu" relacją równoważności jest kuszące, natomiast relacje równoważności określone są na \textit{zbiorach}. Kolekcja obiektów
  kategorii na ogół jest za duża do bycia zbiorem.}}.

  [4] Pokaż, że:
  \begin{enumerate}
    \item dla każdego obiektu $A$ istnieje izomorfizm z $A$ do $A$.
    \item jeśli istnieje izomorfizm z $A$ do $B$, to istnieje też izomorfizm z $B$ do $A$.
    \item jeśli istnieje izomorfizm z $A$ do $B$ oraz z $B$ do $C$, to istnieje też izomorfizm z $A$ do $C$.
  \end{enumerate}
\end{prob}

\begin{prob}
  \emph{To zadanie ma przekonać Cię, że znasz już sporo własności izomorfizmów.}
  [3]
  \begin{enumerate}
  \item Pokaż, że każdy izomorfizm jest monomorfizmem oraz epimorfizmem.
  \item Skonstruuj kategorię, dowodzącą, że twierdzenie odwrotne nie jest prawdziwe (czyli, że istnieje morfizm będący epimorfizmem i monomorfizmem, natomiast nie izomorfizmem).
  \end{enumerate}
\end{prob}

\begin{prob}
  [1] Czym są izomorfizmy w kategorii $\Set$?
\end{prob}

\section{Kategoria $\textbf{Set}_*$ i funktory}
\emph{Zbiorem z wyróżnionym punktem} nazywamy parę $(X, x)$, gdzie $X$ jest zbiorem, a $x$ jakimś jego elementem (czyli $x\in X$). Przekształceniem między zbiorami z wyróżnionymi punktami $(X, x)$ oraz $(Y,y)$ (zapisywanym jako $f: (X, x)\to (Y,y)$) nazwiemy funkcję $f: X\to Y$ taką, że $f(x)=y$.

\begin{prob}
  [5] Pokaż, że zbiory z wyróżnionym punktem wraz z wymienionymi funkcjami tworzą kategorię. Nazywamy ją $\Setstar$.
\end{prob}

\emph{Funktorem} nazywamy regułę $\zeta$,
\footnote{Z przyczyn teoriomnogościowych nie jest to funkcja (trudno określić iloczyn kartezjański dwóch kolekcji, a co dopiero wybrać z niej (pod)zbiór).}
która każdemu obiektowi $A$ w kategorii $\mathcal C$ przypisuje obiekt $\zeta(A)$ w kategorii $\mathcal D$, a każdemu morfizmowi $f\in \Hom(A,B)$ w kategorii $\mathcal C$ przypisuje
morfizm $\zeta(f)\in \Hom(\zeta(A), \zeta(B))$ w kategorii $\mathcal D$ zachowując złożenie
\footnote{Tak naprawdę w kategorii $\mathcal C$ mamy jedno złożenie - $\circ_{\mathcal C}$, a w kategorii jakieś zupełnie inne - $\circ_{\mathcal D}$. Dlatego też
formalista mógłby napisać $$\zeta(f\circ_{\mathcal C} g)=\zeta(f)\circ_{\mathcal D} \zeta(g).$$ My nie będziemy tego robić.}
:

$$\zeta(f\circ g)=\zeta(f)\circ \zeta(g).$$

\begin{prob}
  [3] Rozważmy kategorie $\Setstar$ oraz $\Set$ i wprowadźmy funktor\footnote{Będąc bardzo formalnymi, powinno się napisać $\zeta((X, x))$, bo $(X,x)$ jest przekształcanym
  obiektem, więc powinien stać w dodatkowych nawiasach.}
  $\zeta(X, x) = X$, który zamienia zbiory z wyróżnionym punktem na zbiory.
  \begin{enumerate}
    \item Na co przechodzi funkcja $f: (X,x)\to (Y,y)$?
    \item Pokaż, że $\zeta$ jest funktorem (znaczy, że zachowuje złożenie).
  \end{enumerate}
\end{prob}

\section{Grupy}
\emph{Grupa} to para $(G, *)$, gdzie $G$ jest zbiorem, a funkcja $*: G\times G\to G$ (będziemy pisać $g*h$ lub $gh$ zamiast $*(g,h)$) ma następujące własności:
\begin{enumerate}
  \item istnieje element $e\in G$ taki, że $ge=eg=g$ dla każdego $g\in G$. Będziemy nazywać go \emph{elementem neutralnym}.
  \item dla każdego $g\in G$ istnieje element $\tilde g\in G$ taki, że $g\tilde g=\tilde g g=e.$ Zazwyczaj piszemy $g^{-1}$ zamiast $\tilde g$ oraz nazywamy go
  \emph{elementem odwrotnym do $g$}.
  \item dla dowolnych $f, g, h\in G$ zachodzi równość $(fg)h=f(gh)$.
\end{enumerate}

\begin{prob}
  [3] Rozstrzygnij czy następujące zbiory z działaniami są grupami:
  \begin{enumerate}
    \item $(\mathbb Z, +)$
    \item $(\mathbb Z, \cdot)$
    \item $(\mathbb Q, \cdot)$
    \item $(\mathbb Q^*, \cdot)$, gdzie $\mathbb Q^*=\mathbb Q\setminus \{0\}$
    \item $(\mathbb Q^*, +)$
    \item $(\sigma_n, \circ)$, gdzie $\sigma_n$ oznacza zbiór permutacji
    \footnote{Permutacja na zbiorze $X$ to bijekcja $X\to X$.}
     zbioru $n$-elementowego, a $\circ$ jest zwykłym złożeniem funkcji
  \end{enumerate}
\end{prob}

\begin{prob}
  [4] Dlaczego:
    \begin{enumerate}
      \item istnieje \emph{dokładnie jeden} element neutralny?
      \item dla każdego elementu $g$ istnieje \emph{dokładnie jeden} element odwrotny?
    \end{enumerate}
\end{prob}

\subsection{Homomorfizmy}
Jeśli $G$ oraz $H$ są grupami, to funkcję $f: G\to H$ nazywamy \emph{homomorfizmem} jeśli $f(gh)=f(g)f(h)$ dla każdych $g,h\in G$.
\footnote{Zauważ, że nie piszemy już symbolu "mnożenia"}

\begin{prob}
  [4] Pokaż, że jeśli $f: G\to H$ jest homomorfizmem, to $f(g^{-1})=f(g)^{-1}$ oraz $f(e)=\epsilon$, gdzie $e\in G, \epsilon\in H$ oznaczają elementy neutralne odpowiednich grup.
\end{prob}

\begin{prob}
  [3] Pokaż, że grupy wraz z homomorfizmami tworzą kategorię. Nazywamy ją $\Grp$.
\end{prob}

\begin{prob}
  [2] Wykaż, że izomorfizmy w $\Grp$ to dokładnie bijektywne homomorfizmy.
\end{prob}

\begin{prob}
  [8] Ile jest, z dokładnością do izomorfizmu, grup:
  \begin{enumerate}
    \item o dwóch elementach?
    \item o trzech elementach?
    \item o czterech elementach?
  \end{enumerate}
  Podpowiedź: pomyśl nad ograniczeniami jakie narzuca łączność mnożenia oraz istnienie elementu odwrotnego.
\end{prob}

\section{Topologia}
\subsection{Przeciwobraz funkcji}
Niech $f: X\to Y$ będzie funkcją. Definiujemy pojęcia:
\begin{enumerate}
  \item \emph{obrazu} podzbioru: jeśli $A\subseteq X$, to
    $$f(A) := \{y\in Y : \text{istnieje } x\in X \text{ takie, że } f(x)=y\}$$
  \item \emph{przeciwobrazu} podzbioru: jeśli $B\subseteq Y$, to
    $$f^{-1}(B):=\{x\in X : f(x)\in B\}$$
\end{enumerate}

\begin{prob}
  \emph{To zadanie pokazuje dlaczego później będziemy korzystać z przeciwobrazu.}
  [12] Niech $f:X\to Y$, oraz $A, B\subseteq X$ i $C, D\subseteq Y$. Pokaż, że:
  \begin{enumerate}
    \item $f(A\cap B)\subseteq f(A)\cap f(B)$.
    \item Podaj przykład, dowodzący, że zawierania nie można zastąpić znakiem równości.
    \item $f(A\cup B)=f(A)\cup f(B)$
    \item $f^{-1}(C\cap D)=f^{-1}(C)\cap f^{-1}(D)$
    \item $f^{-1}(C\cup D)=f^{-1}(C)\cup f^{-1}(D)$
    \item $f^{-1}(Y\setminus C)=X\setminus f^{-1}(C)$
  \end{enumerate}
\end{prob}


\subsection{Przestrzeń topologiczna}
Niech $X$ będzie zbiorem. Jego \emph{zbiorem potęgowym} $\mathcal P(X)$ nazywamy zbiór wszystkich jego podzbiorów
\footnote{Jego istnieje gwarantuje aksjomatyka teorii mnogości.}
: $\mathcal P(X)=\{A: A\subseteq X\}$.

\begin{prob}
  [1] Niech $X$ będzie zbiorem o skończonej liczbie elementów $n$. Ile elementów ma zbiór $\mathcal P(n)$?
\end{prob}

\emph{Przestrzenią topologiczną} nazywamy parę $(X, \mathcal T)$, gdzie $X$ jest zbiorem, a $\mathcal T\subseteq \mathcal P(X)$ ma następujące własności:
\begin{enumerate}
  \item $\varnothing, X\in \mathcal T$
  \item jeśli $A_i\in \mathcal T$ dla $i\in I$, to ich suma też należy do $\mathcal T$:
    $$\bigcup_{i\in I}A_i\in \mathcal T$$
  \item jeśli $A,B\in \mathcal T$, to $A\cap B\in\mathcal T$
\end{enumerate}
Zbiór $\mathcal T$ nazywamy \emph{topologią}, a jego elementy - \emph{zbiorami otwartymi}.

\begin{prob}
  [8] Niech $X$ będzie nieskończonym zbiorem. Pokaż, że następujące zbiory są topologiami na $X$:
  \begin{enumerate}
    \item $\mathcal T=\{\varnothing, X\}$ (tak zwana \emph{topologia trywialna})
    \item $\mathcal T=\mathcal P(X)$ (\emph{topologia dyskretna})
    \item $\mathcal T=\{\varnothing\}\cup \{X\setminus A : A \text{ jest skończonym podzbiorem } X\}$ (\textit{topologia koskończona\footnote{Chyba}})
  \end{enumerate}
\end{prob}

\begin{prob}
  \emph{Zazwyczaj pisząc "przestrzeń topologiczna $\mathbb R$" ma się na myśli $\mathbb R$ wyposażoną w pewną określoną topologię...}


  [5] Mówimy, że $A\subseteq \mathbb R$ jest otwarty jeśli dla każdego $x\in A$ istnieje odcinek otwarty $(a_x, b_x)$ taki, że $x\in (a_x, b_x)\subseteq A$.
  \begin{enumerate}
    \item Pokaż, że zbiory otwarte według powyższej definicji spełniają wszystkie aksjomaty topologii.
    \item Pokaż, że każdy zbiór otwarty może być zapisany jako suma pewnej rodziny otwartych odcinków - to znaczy, że jeśli $A$ jest zbiorem otwartym, to istnieją takie odcinki
    $(a_i,b_i),$ $i\in I$, że:
    $$A=\bigcup_i ~(a_i, b_i)$$
  \end{enumerate}
\end{prob}



\subsection{Funkcje ciągłe}
Niech $(X, \mathcal T)$ oraz $(Y, \tau)$ będą przestrzeniami topologicznymi. Funkcję $f: X\to Y$ taką, że $f^{-1}(B)\in \mathcal T$ dla każdego $B\in \tau$ nazywamy \emph{funkcją ciągłą} i oznaczamy $f: (X, \mathcal T)\to (Y, \tau)$.

\begin{prob}
  [5] Pokaż, że funkcja identycznościowa $\Id_X: (X, \mathcal T)\to (X, \mathcal T)$ jest ciągła.
\end{prob}

\begin{prob}
  [5] Pokaż, że funkcja stała $f: (X, \mathcal T)\to (Y, \tau),~f(x)=c$ jest ciągła.
\end{prob}

\begin{prob}
  [3] Pokaż, że funkcja $f: \mathbb R\to \mathbb R$ dana wzorem $f(x)=x^2$ jest ciągła. Podpowiedź: jak można zapisać każdy zbiór otwarty w $\mathbb R$? Jakie własności ma przeciwobraz?
\end{prob}

\subsection{Kategoria $\Top$}

\begin{prob}
  [5] Pokaż, że przestrzenie topologiczne wraz z funkcjami ciągłymi tworzą kategorię. Nazywamy ją $\Top.$
\end{prob}

\begin{prob}
  [3] Dlaczego:
  \begin{enumerate}
    \item monomorfizmy w $\Top$ to dokładnie (ciągłe) iniekcje?
    \item epimorfizmy to dokładnie (ciągłe) surjekcje?
    \item izomorfizmy to dokładnie (ciągłe) bijekcje? W tej kategorii nazywamy je \emph{homeomorfizmami}.
  \end{enumerate}
\end{prob}

Powodzenia!
\end{document}
