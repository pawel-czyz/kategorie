\documentclass{article}

\usepackage{graphicx}
\usepackage[bottom]{footmisc}
\usepackage{listings}
\usepackage{amsmath}
\usepackage{amssymb}
\usepackage{polski}
\usepackage[T1]{fontenc}
\usepackage[utf8]{inputenc}
\usepackage{mathtools}
\usepackage{tikz-cd}

\newcounter{itemnum}
\newenvironment{prob}
{\stepcounter{itemnum}
\paragraph*{Zadanie \arabic{itemnum}.}}
{~\newline}

\newcounter{itemnum2}
\newenvironment{sol}
{\stepcounter{itemnum2}
\paragraph*{Rozwiązanie zadania \arabic{itemnum2}.}}
{~\newline}

\newcommand{\points}[1]{\textbf{[#1]}}

\newcommand{\Z}{\mathbb{Z}}
\newcommand{\R}{\mathbb{R}}

\newcommand{\Hom}{\text{Hom}}
\newcommand{\Id}{\text{Id}}
\newcommand{\Set}{\textbf{Set}}
\newcommand{\Setstar}{\textbf{Set}_*}
\newcommand{\Top}{\textbf{Top}}
\newcommand{\Grp}{\textbf{Grp}}
\newcommand{\Example}{\textbf{Example}}

\begin{document}
\section{Kategorie}
\begin{prob}
  \points{8} Niech obiektami będą zbiory, morfizmami funkcje, a operacją $\circ$ złożenie funkcji\footnote{Cóż za niezwykły zbieg okoliczności - to ten sam symbol!}. Pokaż, że
  jest to kategoria dowodząc, że:
  \begin{enumerate}
    \item dla dowolnych zbiorów $A, B$, wszystkie funkcje z $A$ do $B$ tworzą zbiór (nazywany $\Hom(A,B)$).
    \item dla dowolnych trzech funkcji $f:A\to B, g:B\to C, h:C\to D$ mamy równość: $(h\circ g)\circ f = h\circ(g\circ f).$
    \item dla każdego zbioru $X$ znajdź funkcję\footnote{Ciekawe jak się może nazywać...} $\Id_X$ taką, że dla dowolnej $f:A\to B$ mamy $\Id_B\circ f=f=f\circ\Id_A$.
  \end{enumerate}
  Kategorię tę nazywamy $\Set$.
\end{prob}
\begin{sol}
  \begin{enumerate}
    \item \points{3} Rozważmy zbiory $A, B$. Każda funkcja z $A$ do $B$ to z definicji pewien podzbiór $A\times B$, zatem wszystkie funkcje z $A$ do $B$ to pewien podzbiór zbioru potęgowego $A\times B$.
    \item \points{3} Rozważmy funkcje podane $f,~g,~h$ takie jak w treści zadania. Wówczas dla każdego $a\in A$ zachodzi równość $$((h\circ g)\circ f)(a)=h(g(f(a))=(h\circ (g\circ f))(a),$$ która pociąga tezę zadania.
    \item \points{2} Funkcja identycznościowe postaci $\Id_X=\{(x,x)\in X\times X : x\in X\}$ są tymi szukanymi, ponieważ dla każdego $a\in A$ mamy równości
    $$\Id_B(f(a))=f(a)=f(\Id_A(a)).$$
  \end{enumerate}
\end{sol}

\section{Morfizmy}
\subsection{Epimorfizmy}

\begin{prob}
  \points{4} Pokaż, że jeśli $f:A\to B$ oraz $g: B\to C$ są epimorfizmami, to $g\circ f$ też jest epimorfizmem.
\end{prob}
\begin{sol}
  \points{4} Rozważmy dowolne morfizmy $\alpha, \alpha' : C\to D$ takie, że $\alpha\circ g\circ f=\alpha'\circ g\circ f$. Ponieważ $f$ jest epimorfizmem otrzymujemy z tego równość:
  $\alpha\circ g=\alpha'\circ g$. Ponieważ $g$ jest epimorfizmem otrzymujemy $\alpha=\alpha'$, co należało wykazać.
\end{sol}

\begin{prob}
  \points{4} Pokaż, że w kategorii $\Set$ epimorfizmy to dokładnie surjekcje (każdy epimorfizm jest surjekcją, a każda surjekcja epimorfizmem).
\end{prob}
\begin{sol}
  \points{2} \textit{Każda surjekcja to epimorfizm:} Niech $f: A\to B$ będzie surjekcją. Chcemy pokazać, że dla dowolnych funkcji $g,g': B\to C$ takich, że $g\circ f=g\circ g'$, wynika, że $g=g'$.
  Przypuśćmy, że istnieje takie $b\in B$, że $g(b)\neq g'(b)$. Ponieważ $f$ jest surjekcją, istnieje takie $a\in A$, że $f(a)=b$. Zatem otrzymujemy:
  $g(b)=g(f(a))=(g\circ f)(a)=(g'\circ f)(a)=g'(f(a))=g'(b)$, co stoi w sprzeczności z przypuszczeniem.

  \points{2} \textit{Każdy epimorfizm to surjekcja:} Niech $f: A\to B$ będzie epimorfizmem niebędącym surjekcją. Możemy zatem określić funkcję:
  $g: B\ \to \{0, 1\},$
  taką, że $g(a)=1$ dla $a$ leżącego w obrazie funkcji $f$ oraz $g(b)=0$ dla $b$ nieleżących w obrazie funkcji $f$. Niech teraz $g': B\to \{0,1\}$ będzie funkcją stałą, równą 1.
  Wówczas $g\circ f=g'\circ f$, lecz $g\neq g'$, co pokazuje, że epimorfizm niebędący surjekcją nie może istnieć.
\end{sol}

\begin{prob}
  \points{5} Skonstruuj kategorię, której obiektami są pewne zbiory (tak więc pojęcie surjekcji ma sens), natomiast istnieje epimorfizm niebędący surjekcją.
\end{prob}
\begin{sol}
  \points{5} Kategorią $\Example$ nazwijmy kategorię, której obiektami są zbiory $A=\{0, 1\}$ oraz $B=\{5, 10\}$, a morfizmami $\Id_A, \Id_B$ oraz $f: A\to B, f(a)=5$ dla każdego $a\in A$.
  Wówczas $f$ nie jest surjekcją, lecz jest epimorfizmem.
\end{sol}

\subsection{Monomorfizmy}
\begin{prob}
  \points{1} Pokaż, że jeśli $f:A\to B$ oraz $g: B\to C$ są monomorfizmami, to $g\circ f$ też jest monomorfizmem.
\end{prob}
\begin{sol}
  \points{1} Dowód przebiega dokładnie tak samo jak dla epimorfizmów, poza tym, że skracamy funkcje poczynając od tej najbardziej z lewej.
\end{sol}

\begin{prob}
  \points{1} Pokaż, że w kategorii $\Set$ monomorfizmy to dokładnie iniekcje (każdy monomorfizm jest iniekcją, a każda iniekcja monomorfizmem).
\end{prob}
\begin{sol}
  \points{1} Dowód przebiega analogicznie do epimorfizmów i surjekcji, natomiast wprowadzamy funkcje ze zbioru jednoelementowego $\{0\}$.
\end{sol}

\begin{prob}
  \points{1} Skonstruuj kategorię, której obiektami są pewne zbiory (tak więc pojęcie iniekcji ma sens), natomiast istnieje monomorfizm niebędący iniekcją.
\end{prob}
\begin{sol}
  \points{1} Kategoria $\Example$ ponownie okazuje się poprawnym przykładem.
\end{sol}

\subsection{Izomorfizmy}
\begin{prob}
  \emph{Będziemy wykorzystywać izomorfizmy do utożsamiania pewnych przestrzeni. W tym celu przydadzą nam się własności podobne do tych występujących w definicji relacji równoważności \footnote{Nazwanie "dokładności do izomorfizmu" relacją równoważności jest kuszące, natomiast relacje równoważności określone są na \textit{zbiorach}. Kolekcja obiektów
  kategorii na ogół jest za duża do bycia zbiorem.}}.

  \points{4} Pokaż, że:
  \begin{enumerate}
    \item dla każdego obiektu $A$ istnieje izomorfizm z $A$ do $A$.
    \item jeśli istnieje izomorfizm z $A$ do $B$, to istnieje też izomorfizm z $B$ do $A$.
    \item jeśli istnieje izomorfizm z $A$ do $B$ oraz z $B$ do $C$, to istnieje też izomorfizm z $A$ do $C$.
  \end{enumerate}
\end{prob}
\begin{sol}
  \begin{enumerate}
    \item \points{1} Dla każdego obiektu $A$ istnieje morfizm $\Id_A$, który jest izomorfizmem.
    \item \points{1} Wynika to wprost z definicji izomorfizmu.
    \item \points{2} Niech $f$ będzie izomorfizmem z $A$ do $B$, a $g$ izomorfizmem z $B$ do $C$. Przez $m'$ oznaczmy izomorfizm "odwrotny" do izomorfizmu $m$. Wówczas
      $g\circ f$ jest izomorfizmem z $A$ do $C$ ponieważ $(g\circ f)\circ (f'\circ g')=\Id_C$ oraz $(f'\circ g')\circ (g\circ f)=\Id_A$.
  \end{enumerate}
\end{sol}

\begin{prob}
  \emph{To zadanie ma przekonać Cię, że znasz już sporo własności izomorfizmów.}
  \points{3}
  \begin{enumerate}
  \item Pokaż, że każdy izomorfizm jest monomorfizmem oraz epimorfizmem.
  \item Skonstruuj kategorię, dowodzącą, że twierdzenie odwrotne nie jest prawdziwe (czyli, że istnieje morfizm będący epimorfizmem i monomorfizmem, natomiast nie izomorfizmem).
  \end{enumerate}
\end{prob}
\begin{sol}
  \begin{enumerate}
    \item \points{2} Niech $f: A\to B$ będzie izomorfizmem, czyli istnieje $g:B\to A$, które odpowiednio złożone z $f$ daje identyczności na $A$ i $B$. Aby pokazać, że $f$ jest epimorfizmem
    rozważmy morfizmy $\alpha, \alpha'$ takie, że $\alpha \circ f=\alpha' \circ f$. Składając obydwie strony równości z $g$ otrzymujemy $\alpha \circ \Id_B = \alpha'\circ Id_B$,
    co pokazuje, że $\alpha=\alpha'$. Dowód, że $f$ jest monomorfizmem przebiega tak samo, z tym, że dokładamy $g$ od lewej strony.
    \item \points{1} Kategoria $\Example$ raz jeszcze.
  \end{enumerate}
\end{sol}

\begin{prob}
  \points{1} Czym są izomorfizmy w kategorii $\Set$?
\end{prob}
\begin{sol}
  \points{1} Wiemy, że każdy izomorfizm to monomorfizm i epimorfizm, czyli w $\Set$ każdy izomorfizm musi być iniekcją i surjekcją jednocześnie. Zatem wszystkimi kandydatami na izomorfizmy
  są bijekcje. Dla każdej bijekcji istnieje funkcja odwrotna, co pokazuje, że \textit{bijekcje to dokładnie izomorfizmy w kategorii $\Set.$}
\end{sol}

\section{Kategoria $\textbf{Set}_*$ i funktory}
\begin{prob}
  \points{5} Pokaż, że zbiory z wyróżnionym punktem wraz z wymienionymi funkcjami tworzą kategorię. Nazywamy ją $\Setstar$.
\end{prob}
\begin{sol}
\begin{enumerate}
  \item \points{2} Rozważmy dwa obiekty: $(X, x)$ oraz $(Y,y)$ wszystkie funkcje z $X$ do $Y$ zamieniające $x$ na $y$ to podzbiór zbioru funkcji z $X$ do $Y$ (dowód, że wszystkie funkcje z $X$
  do $Y$ tworzą zbiór został przeprowadzony w zadaniu 1.)
  \item \points{2} Rozważmy morfizmy $f: (A,a)\to (B,b), ~g: (B,b)\to (C,c)$. Wówczas $g\circ f: A\to C$ oraz $g(f(a))=g(b)=c$, więc morfizmy składają się poprawnie
  \item \points{1} Dla każdego obiektu $(X,x)$ funkcja identycznościowa $\Id_X$ jest dobrym morfizmem.
\end{enumerate}
\end{sol}

\begin{prob}
  \points{3} Rozważmy kategorie $\Setstar$ oraz $\Set$ i wprowadźmy funktor\footnote{Będąc bardzo formalnymi, powinno się napisać $\zeta((X, x))$, bo $(X,x)$ jest przekształcanym
  obiektem, więc powinien stać w dodatkowych nawiasach.}
  $\zeta(X, x) = X$, który zamienia zbiory z wyróżnionym punktem na zbiory.
  \begin{enumerate}
    \item Na co przechodzi funkcja $f: (X,x)\to (Y,y)$?
    \item Pokaż, że $\zeta$ jest funktorem (znaczy, że zachowuje złożenie).
  \end{enumerate}
\end{prob}
\begin{sol}
  \begin{enumerate}
    \item \points{1} Funkcja $f$ jako obiekt matematyczny (czyli w zasadzie jako pewien podzbiór $X\times Y$) się nie zmienia - natomiast teraz jest traktowana jako morfizm w innej kategorii
    \item \points{2} Dla dowolnego obiektu $(X,x)$ w $\Setstar$ mamy $\zeta(\Id_{(X,x)})=\Id_X$ oraz dla dowolnych morfizmów $\zeta(f\circ g)=\zeta(f)\circ \zeta(g)$, co wynika z powyższej
      obserwacji.
  \end{enumerate}
\end{sol}

\section{Grupy}
\emph{Grupa} to para $(G, *)$, gdzie $G$ jest zbiorem, a funkcja $*: G\times G\to G$ (będziemy pisać $g*h$ lub $gh$ zamiast $*(g,h)$) ma następujące własności:
\begin{enumerate}
  \item istnieje element $e\in G$ taki, że $ge=eg=g$ dla każdego $g\in G$. Będziemy nazywać go \emph{elementem neutralnym}.
  \item dla każdego $g\in G$ istnieje element $\tilde g\in G$ taki, że $g\tilde g=\tilde g g=e.$ Zazwyczaj piszemy $g^{-1}$ zamiast $\tilde g$ oraz nazywamy go
  \emph{elementem odwrotnym do $g$}.
  \item dla dowolnych $f, g, h\in G$ zachodzi równość $(fg)h=f(gh)$.
\end{enumerate}

\begin{prob}
  \points{3} Rozstrzygnij czy następujące zbiory z działaniami są grupami:
  \begin{enumerate}
    \item $(\mathbb Z, +)$
    \item $(\mathbb Z, \circ)$
    \item $(\mathbb Q, \circ)$
    \item $(\mathbb Q^*, \circ)$, gdzie $\mathbb Q^*=\mathbb Q\setminus \{0\}$
    \item $(\mathbb Q^*, +)$
    \item $(\sigma_n, \circ)$, gdzie $\sigma_n$ oznacza zbiór permutacji
    \footnote{Permutacja zbioru $X$ to bijekcja $X\to X$.}
     zbioru $n$-elementowego, a $\circ$ jest zwykłym złożeniem funkcji
  \end{enumerate}
\end{prob}
\begin{sol}
  \begin{enumerate}
    \item \points{0.5} Tak - 0 jest elementem neutralnym, a elementem odwrotnym do $a\in \Z$ jest $-a$. Ponadto dla dowolnych $a,b,c$ mamy $(a+b)+c=a+(b+c).$
    \item \points{0.5} Nie - w każdej grupie element neutralny jest unikatowy (zobacz następne zadanie), tutaj musiałaby to być jedynka. Wówczas liczba 2 nie ma elementu odwrotnego
    (liczba $1/2$ nie jest liczbą całkowitą)
    \item \points{0.5} Nie - dla każdej liczby $x$ mamy $0\circ x=0$, więc 0 musiałoby być elementem neutralnym mnożenia. A nie jest bo $0\circ5\neq 5$ (lub też dlatego, że jest nim jedynka, a
    element neutralny jest unikatowy)
    \item \points{0.5} Tak - elementem neutralnym jest jedynka, elementem odwrotnym jest liczba odwrotna, a mnożenie jest łączne.
    \item \points{0.5} Nie - elementem neutralnym musiałoby być 0, którego nie ma w tym zbiorze
    \item \points{0.5} Tak - funkcja identycznościowa jest permutacją i elementem neutralnym, do każdej permutacji istnieje permutacja odwrotna, a składanie funkcji jest łączne.
  \end{enumerate}
\end{sol}

\begin{prob}
  \points{4} Dlaczego:
    \begin{enumerate}
      \item istnieje \emph{dokładnie jeden} element neutralny?
      \item dla każdego elementu $g$ istnieje \emph{dokładnie jeden} element odwrotny?
    \end{enumerate}
\end{prob}
\begin{sol}
  \begin{enumerate}
    \item \points{2} Jeśli $e$ oraz $e'$ są elementami neutralnymi to $e=ee'=e'$.
    \item \points{2} Niech $h$ i $h'$ będą elementami odwrotnymi do $g$. Zatem:
    $$h=he=h(gh')=(hg)h'=eh'=h'.$$
  \end{enumerate}
\end{sol}

\subsection{Homomorfizmy}
\begin{prob}
  \points{4} Pokaż, że jeśli $f: G\to H$ jest homomorfizmem, to $f(g^{-1})=f(g)^{-1}$ oraz $f(e)=\epsilon$, gdzie $e\in G, \epsilon\in H$ oznaczają elementy neutralne odpowiednich grup.
\end{prob}
\begin{sol}
  \points{2} Dla dowolnego $g\in G$ mamy $f(g)=f(ge)=f(g)f(e)$ i analogicznie $f(g)=f(e)f(g)$. Zatem $f(e)$ jest elementem neutralnym $H$.
  \points{2} Wówczas: $f(g^{-1})f(g)=f(g^{-1}g)=f(e)=\epsilon,$ więc $f(g^{-1})=f(g)^{-1}.$
\end{sol}

\begin{prob}
  \points{3} Pokaż, że grupy wraz z homomorfizmami tworzą kategorię. Nazywamy ją $\Grp$.
\end{prob}
\begin{sol}
 \points{1} Dla dowolnych homomorfizmów $f: F\to G$ i $g: G\to H$, złożenie również jest homomorfizmem, gdyż $(f\circ g)(ab) = f(g(ab))) = f(g(a)g(b)) = f(g(a))f(g(b)) = (f\circ g (a)) (f\circ g(b))$ dla dowolnych $a, b \in F$. \points{1} Składanie funkcji jest łączne \points{1} Morfizm $\text{Id}_G: G \ni g \to g \in G$ spełnia $\text{Id}_G \circ f  = f$ dla wszystkich $f \in \text{Hom}(F, G)$ oraz $g\circ \text{Id}_G  = g$ dla wszystkich $g \in \text{Hom}(G, H)$.
\end{sol}


\begin{prob}
  \points{2} Wykaż, że izomorfizmy w $\Grp$ to dokładnie bijektywne homomorfizmy.
\end{prob}
\begin{sol}
 \points{1} Niech $f: G\to H$ będzie dowolnym bijektywnym homomorfizmem. Niech $g: H\ni h\to f^{-1}(h) \in G$. Funkcja $g$ jest dobrze określona, gdyż $f$ jest bijekcją. Mamy $g\circ f = \text{Id}_G$ oraz $f\circ g = \text{Id}_H$. Dla dowolnych $a, b \in H$ zachodzi $g(ab) = g(f(a')f(b')) = g(f(a'b')) = a'b' = g(a)g(b)$, gdzie $a' = g(a), b' = g(b)$, więc $g$ jest morfizmem, a więc także izomorfizmem. \points{1} Niech teraz $f: G\to H$ będzie dowolnym izomorfizmem. Odwracalność daje daje bijektywność, a ponieważ $f$ jest morfizmem jest też homomorfizmem.
\end{sol}


\begin{prob}
  \points{8} Ile jest, z dokładnością do izomorfizmu\footnote{To znaczy, że nie rozróżniamy grup, które są izomorficzne. Na przykład w kategorii $\Set$ istnieje tylko jeden zbiór dwunastoelementowy, licząc z dokładnością do izomorfizmu,
  czyli bijekcji.}, grup:
  \begin{enumerate}
    \item o dwóch elementach?
    \item o trzech elementach?
    \item o czterech elementach?
  \end{enumerate}
  Podpowiedź: pomyśl nad ograniczeniami jakie narzuca łączność mnożenia oraz istnienie elementu odwrotnego.
\end{prob}
\begin{sol}
  \begin{enumerate}
    \item \points{2} Jeden element to element neutralny $e$. Drugi oznaczmy przez $g$. Wówczas $g^2=e$ jako, że $ge=eg=g\neq e$, a $g$ musi mieć element odwrotny. Zatem mamy dokładnie jedną
    taką grupę, powstałą według reguł mnożenia: $e^2=g^2=e$, $ge=eg=g$. Można zatem powiedzieć, że grupa $\Z_2$ (czyli liczby $0,\, 1$ z dodawaniem modulo 2) to \emph{jedyna} grupa
    dwuelementowa.
    \item \points{3} Oznaczmy elementy grupy przez $e,b,c$. Jeśli $bc=b$, to mnożąc od lewej strony tę równość przez element odwrotny do $b$ otrzymalibyśmy $c=e$. Analogicznie $bc$ nie może
      być równe $c$. Zatem jedyną możliwością jest $bc=e$. Stąd otrzymujemy też $cb=e$. Musimy mieć $b^2=b$ lub $b^2=c$. W pierwszym przypadku mamy $b=e$, więc $b^2=c$. Analogicznie
      $c^2=b$. Znamy jednoznacznie wyniki mnożenia każdych dwóch elementów tej grupy, jeśli więc mnożenie okaże się łączne, to wiemy, że istnieje dokładnie jedna taka grupa. Biorąc
      jako przykład $\Z_3$ widzimy, że istnieje \textit{dokładnie} jedna trzyelementowa grupa.
    \item \points{3} Oznaczmy elementy grupy przez $e,a,b,c$. Bez straty ogólności możemy przyjąć, że $b^2=e$ (parując element z jego elementem odwrotnym widzimy, że któryś musi zostać
      bez pary). Teraz mamy dwie możliwości:
      \begin{enumerate}
        \item $a^2=e$, więc i $c^2=e$. Wówczas $ab$ musi być równe $c$ (czemuś być musi, a elementy $e,a,b$ prowadzą do sprzeczności). Analogicznie znajdujemy pozostałe relacje mnożenia. Taka grupa nazywana jest \textit{grupą Kleina} i może być skonstruowana jako $\Z_2\times \Z_2$, wraz z dodawaniem modulo dwa po współrzędnych.
        \item $e$ i $b$ to wszystkie elementy $g$ takie, że $g^2=e$, czyli $ac=ca=e$. Zastanówmy się ile wynosi $ba$. Nie może to być $e$, ani $a$ ani $b$. Musi zatem to być $c$.
        Analogicznie jednoznacznie znajdujemy $ab$, $cb$, $bc$. Tą grupą okazuje się być $\Z_4$.
      \end{enumerate}
      Zatem są \textit{dwie} grupy czteroelementowe.
  \end{enumerate}
\end{sol}

\section{Topologia}
\subsection{Przeciwobraz funkcji}
Niech $f: X\to Y$ będzie funkcją. Definiujemy pojęcia:
\begin{enumerate}
  \item \emph{obrazu} podzbioru: jeśli $A\subseteq X$, to
    $$f(A) := \{y\in Y : \text{istnieje } x\in X \text{ takie, że } f(x)=y\}$$
  \item \emph{przeciwobrazu} podzbioru: jeśli $B\subseteq Y$, to
    $$f^{-1}(B):=\{x\in X : f(x)\in B\}$$
\end{enumerate}

\begin{prob}
  \emph{To zadanie pokazuje dlaczego później będziemy korzystać z przeciwobrazu.}
  \points{12} Niech $f:X\to Y$, oraz $A, B\subseteq X$ i $C, D\subseteq Y$. Pokaż, że:
  \begin{enumerate}
    \item $f(A\cap B)\subseteq f(A)\cap f(B)$.
    \item Podaj przykład, dowodzący, że zawierania nie można zastąpić znakiem równości.
    \item $f(A\cup B)=f(A)\cup f(B)$
    \item $f^{-1}(C\cap D)=f^{-1}(C)\cap f^{-1}(D)$
    \item $f^{-1}(C\cup D)=f^{-1}(C)\cup f^{-1}(D)$
    \item $f^{-1}(Y\setminus C)=X\setminus f^{-1}(C)$
  \end{enumerate}
\end{prob}
\begin{sol}
  \begin{enumerate}
    \item \points{2} $f(A\cap B)\subseteq A$ oraz $f(A\cap B)\subseteq B$. Zatem $f(A\cap B)\subseteq A\cap B$
    \item \points{2} Niech $f: \Z\to \{1\}$. Wówczas jeśli $P$ to liczby parzyste, a $N$ to liczby
    nieparzyste, mamy $f(P\cap N)=\varnothing\neq \{1\}$
    \item \points{2} $f^{-1}(C\cap D) = \{x\in X : f(x)\in C\cap D\}$, $f^{-1}(C)\cap f^{-1}(D)=\{x\in X : f(x)\in C\}\cap \{x\in X : f(x)\in D\}= \{x\in X : f(x)\in C\cap D\}$
    \item \points{2} Analogicznie jak wyżej.
    \item \points{2} $f^{-1}(Y\setminus C)=\{x\in X : f(x)\notin C\}=X\setminus \{x\in X : f(x)\in C\}=X\setminus f^{-1}(C)$
  \end{enumerate}
\end{sol}

\subsection{Przestrzeń topologiczna}
Niech $X$ będzie zbiorem. Jego \emph{zbiorem potęgowym} $\mathcal P(X)$ nazywamy zbiór wszystkich jego podzbiorów
\footnote{Jego istnieje gwarantuje aksjomatyka teorii mnogości.}
: $\mathcal P(X)=\{A: A\subseteq X\}$.

\begin{prob}
  \points{1} Niech $X$ będzie zbiorem o skończonej liczbie elementów $n$. Ile elementów ma zbiór $\mathcal P(n)$?
\end{prob}
\begin{sol}
  \points{1} Ponumerujmy elementy zbioru liczbami od 1 do $n$. Wówczas możemy skonstruować każdy podzbiór $X$ poprzez wybieranie czy element $i$ ma się znaleźć w danym podzbiorze czy nie.
  Zatem mamy $2\circ 2\dots\circ 2=2^n$ możliwości skonstruowania podzbiorów i taka też jest moc $\mathcal P(X).$
\end{sol}

\begin{prob}
  \points{8} Niech $X$ będzie nieskończonym zbiorem. Pokaż, że następujące zbiory są topologiami na $X$:
  \begin{enumerate}
    \item $\mathcal T=\{\varnothing, X\}$ (tak zwana \emph{topologia trywialna})
    \item $\mathcal T=\mathcal P(X)$ (\emph{topologia dyskretna})
    \item $\mathcal T=\{\varnothing\}\cup \{X\setminus A : A \text{ jest skończonym podzbiorem } X\}$ (\textit{topologia koskończona
    \footnote{Chyba... jakoś to nie brzmi}})
  \end{enumerate}
\end{prob}
\begin{sol}
  \begin{enumerate}
    \item \points{2} Oczywiście zbiór pusty i cała przestrzeń są otwarte, podobnie też ich dowolne sumy i skończone przecięcia (są to albo cała przestrzeń albo zbiór pusty).
    \item \points{2} Zbiór pusty i cała przestrzeń są otwarte, dowolna suma podzbiorów $X$ nadal jest podzbiorem $X$ (a więc jest zbiorem otwartym), podobnie skończone przecięcia.
    \item \points{4} Korzystając z $A=\varnothing$ dowodzimy, że cała przestrzeń jest otwarta; zbiór pusty jest otwarty z definicji tej topologii. Rozważmy teraz zbiory otwarte
    $U_i, i\in I$ oraz ich dopełnienia $U_i^c=X\setminus U_i, i\in I$. Dopełnienie dowolnej sumy zbiorów $U_i$ to iloczyn dowolnie wielu skończonych $U_i^c$, czyli też zbiór skończony. Pokazuje to, że suma dowolnie wielu zbiorów otwartych też jest zbiorem otwartym. Dopełnienie iloczynu $U_i\cap U_j$ to suma dopełnień $U_i^c\cup U_j^c$, czyli zbiór skończony. Zatem iloczyn dwóch zbiorów otwartych jest zbiorem otwartym.
  \end{enumerate}
\end{sol}

\begin{prob}
  \emph{Zazwyczaj pisząc "przestrzeń topologiczna $\mathbb R$" ma się na myśli $\mathbb R$ wyposażoną w pewną określoną topologię...}

  \points{5} Mówimy, że $A\subseteq \mathbb R$ jest otwarty jeśli dla każdego $x\in A$ istnieje odcinek otwarty $(a_x, b_x)$ taki, że $x\in (a_x, b_x)\subseteq A$.
  \begin{enumerate}
    \item Pokaż, że zbiory otwarte według powyższej definicji spełniają wszystkie aksjomaty topologii.
    \item Pokaż, że każdy zbiór otwarty może być zapisany jako suma pewnej rodziny otwartych odcinków - to znaczy, że jeśli $A$ jest zbiorem otwartym, to istnieją takie odcinki
    $(a_i,b_i),$ $i\in I$, że:
    $$A=\bigcup_i ~(a_i, b_i)$$
  \end{enumerate}
\end{prob}
\begin{sol}
  \begin{enumerate}
    \item Sprawdzamy po kolei wszystkie aksjomaty:
      \begin{enumerate}
        \item \points{1} $\varnothing$ nie ma żadnego punktu, więc wokół każdego da się narysować odcinek; wokół każdej liczby rzeczywistej w $\R$ istnieje odcinek zawarty w $\R$
        \item \points{1} Rozważmy rodzinę zbiorów $\{X_i : i\in I\}$ i sumę $X$. Rozważmy dowolny punkt $x\in X$. Wówczas istnieje takie $i$, że $x\in X_i$. Zatem da się narysować odcinek
        wokół $x$ zawarty w $X_i$, a więc i w $X$.
        \item \points{1} Niech $A$ i $B$ będą otwarte. Rozważmy punkt $x\in A\cap B$. Wówczas istnieją odcinki $(x-a,x+a)\subseteq A$, $(x-b, a+b)\subseteq B$. Odcinek $(x-k, x+k)\subseteq
        A\cap B$ jest szukanym otoczeniem $x$ dla $k=\min(a,b).$
      \end{enumerate}

    \item \points{2} Rozważmy dowolny zbiór otwarty $X$ w $\R$. Wokół każdego punktu $x\in X$ istnieje odcinek $O_x$ zawarty w $X$ taki, że $x\in O_x$. Rozważmy sumę
    odcinków $O_x$ dla $x\in X$. Widzimy, że jest nadzbiorem $X$ (bo zawiera każdy punkt $x\in X$) oraz, że jest podzbiorem $X$ (bo żaden z odcinków nie ma punktów spoza $X$), zatem
    jest równy $X$.
  \end{enumerate}
\end{sol}

\subsection{Funkcje ciągłe}
Niech $(X, \mathcal T)$ oraz $(Y, \tau)$ będą przestrzeniami topologicznymi. Funkcję $f: X\to Y$ taką, że $f^{-1}(B)\in \mathcal T$ dla każdego $B\in \tau$ nazywamy \emph{funkcją ciągłą} i oznaczamy $f: (X, \mathcal T)\to (Y, \tau)$.

\begin{prob}
  \points{5} Pokaż, że funkcja identycznościowa $\Id_X: (X, \mathcal T)\to (X, \mathcal T)$ jest ciągła.
\end{prob}
\begin{sol}
  \points{5} Niech $O$ będzie dowolnym zbiorem otwartym. Wówczas przeciwobraz $\Id_X^{-1}(O)=O,$ czyli jest zbiorem otwartym, co pokazuje ciągłość $\Id_X$.
\end{sol}

\begin{prob}
  \points{5} Pokaż, że funkcja stała $f: (X, \mathcal T)\to (Y, \tau),~f(x)=c$ jest ciągła.
\end{prob}
\begin{sol}
  \points{5} Niech $O$ będzie zbiorem otwartym w $Y$. Jeśli $c\in O$, to $f^{-1}(O)=X$ (czyli jest to zbiór otwarty). Jeśli $c\in Y\setminus O$, to $f^{-1}(O)=\varnothing$ (czyli
  jest to zbiór otwarty).
\end{sol}

\begin{prob}
  \points{3} Pokaż, że funkcja $f: \mathbb R\to \mathbb R$ dana wzorem $f(x)=x^2$ jest ciągła. Podpowiedź: jak można zapisać każdy zbiór otwarty w $\mathbb R$? Jakie własności ma przeciwobraz?
\end{prob}
\begin{sol}
  \points{3}
  Każdy zbiór otwarty w $\R$ można zapisać jako sumę przedziałów otwartych. Wiemy też, że przeciwobraz sumy zbiorów to suma przeciwobrazów oraz, że sumy zbiorów otwartych to
  zbiory otwarte. Zatem wystarczy pokazać, że przeciwobraz każdego otwartego przedziału $\R$ jest zbiorem otwartym.

  Rozważmy zatem przedział $(a,b)$. Jego przeciwobraz to:

  \begin{enumerate}
    \item $\varnothing$ jeśli $(a,b)=\varnothing$ (czyli gdy $a\ge b$). Jest to zbiór otwarty.
    \item $\varnothing$ jeśli $b < 0$. Jest to zbiór otwarty.
    \item $(0, \sqrt b)\cup (-\sqrt b, 0)$ jeśli $a < 0 \le b$. Jest to zbiór otwarty.
    \item $(\sqrt a, \sqrt b)\cup (-\sqrt b, -\sqrt a)$ jeśli $0 \le a < b$. Jest to zbiór otwarty.
  \end{enumerate}
\end{sol}

\subsection{Kategoria $\Top$}
\begin{prob}
  \points{5} Pokaż, że przestrzenie topologiczne wraz z funkcjami ciągłymi tworzą kategorię. Nazywamy ją $\Top.$
\end{prob}
\begin{sol}
  \begin{enumerate}
    \item \points{1} Funkcje ciągłe z $X$ do $Y$ są funkcjami z $X$ do $Y$ o pewnej własności, tworzą zatem zbiór.
    \item \points{3} Niech $(X_i, \mathcal T_i), i=1,2,3$ będą przestrzeniami topologicznymi oraz niech $f_i: X_i\to X_{i+1}, i=1,2$ będą funkcjami ciągłymi. Rozważmy dowolny zbiór
    otwarty $O$ w $X_3$. Wówczas $f_2^{-1}(O)$ jest otwarty w $X_2$ (co wynika z ciągłości $f_2$). Z ciągłości $f_1$ otrzymujemy, że zbiór $f_1^{-1}(f_2^{-1}(O))$ jest otwarty w
    $X_1$, co dowodzi, że funkcja $f_2 \circ f_1$ jest funkcją ciągłą z $X_1$ do $X_3$, zatem mamy działające składanie morfizmów.
    \item \points{1} Pokazaliśmy już, że identyczności są ciągłe.
  \end{enumerate}
\end{sol}

\begin{prob}
  \points{3} Dlaczego:
  \begin{enumerate}
    \item monomorfizmy w $\Top$ to dokładnie ciągłe iniekcje?
    \item epimorfizmy to dokładnie ciągłe surjekcje?
    \item izomorfizmy to dokładnie ciągłe bijekcje posiadające ciągłą odwrotność? W tej kategorii nazywamy je \emph{homeomorfizmami}.
  \end{enumerate}
\end{prob}
\begin{sol}
  \points{3} Wystarczy powtorzyć dowody\footnote{Przyznawany jest jeden punkt za każdy dowód.} dla kategorii $\Set$ pamiętając, że należy zadbać by wprowadzane funkcje były ciągłe (czyli np. wyposażyć
  zbiór $\{0,1\}$ w topologię trywialną gdy rozważa się epimorfizmy).
\end{sol}
\end{document}
